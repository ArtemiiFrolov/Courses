\documentclass[11pt]{article}

% ------------------------------------------------------------------------------
% This is all preamble stuff that you don't have to worry about.
% Head down to where it says "Start here"
% ------------------------------------------------------------------------------

\usepackage[margin=.8in,top=1.1in,bottom=1.1in]{geometry} % page layout
\usepackage{amsmath,amsthm,amssymb,amsfonts} % math things
\usepackage{graphicx} % include graphics
\usepackage{fancyhdr} % header customization
\usepackage{titlesec} % help with section naming

% naming sections
\titleformat{\section}{\bf}{Problem \thesection}{0.5em}{}
\newcommand{\exercise}{\section{}}

% headers
\pagestyle{fancy} 
\fancyhf{} % clear all
\fancyhead[L]{\sffamily\small Machine Learning 1 --- Homework}
\fancyhead[R]{\sffamily\small Page \thepage}
\renewcommand{\headrulewidth}{0.2pt}
\renewcommand{\footrulewidth}{0.2pt}
\markright{\hrulefill\quad}

\newcommand{\hwhead}[4]{
\begin{center}
\sffamily\large\bfseries Machine Learning Worksheet #1
\vspace{2mm} 
\normalfont

#2 -- #3 -- \texttt{#4}
\end{center}
\vspace{6mm} \hrule \vspace{4mm}
}

% ------------------------------------------------------------------------------
% Start here -- Fill in your name, imat and email
% ------------------------------------------------------------------------------

\newcommand{\name}{Artemii Frolov} %
\newcommand{\imat}{03681119} %
\newcommand{\email}{ga83cag@mytum.de} %

\begin{document}

% ------------------------------------------------------------------------------
% Change xx (and only xx) to the current sheet number
% ------------------------------------------------------------------------------
\hwhead{xx}{\name}{\imat}{\email}

% ------------------------------------------------------------------------------
% Fill in your solutions
% ------------------------------------------------------------------------------

\exercise % each new exercise begins with this command
If $w^Tx_i+w_0>0$ for each $x_i$, then if we multiply each inequality by $\alpha_i>0, \sum_i \alpha_i=1 $ and sum all of them, then 
$$\sum_i^N \alpha_i (w^Tx_i+w_0)>0$$
$$\sum_i^N \alpha_i w^Tx_i+\sum_i^N \alpha_i w_0>0$$
$$w^T\sum_i^N \alpha_i x_i+  w_0 \sum_i^N \alpha_i>0$$
$$w^T\sum_i^N \alpha_i x_i+  w_0 >0$$
That means that for each point $z_1$, that belongs $X$ and can be declared as $\sum_i^N \alpha_i x_i$ (as part of $coX$):
$$w^Tz_1 +  w_0 >0$$
If $w^Ty_i+w_0<0$ for each $y_i$, then if we multiply each inequality by $\beta_i>0, \sum_i \beta_i=1 $  and sum all of them, then 
$$\sum_i^M \beta_i (w^Ty_i+w_0)<0$$
$$\sum_i^M \beta_i w^Ty_i+\sum_i^M \beta_i w_0<0$$
$$w^T\sum_i^M \beta_i y_i+  w_0 \sum_i^M \beta_i<0$$
$$w^T\sum_i^M \beta_i y_i+  w_0 <0$$
That means that for each point $z_2$, that belongs $Y$ and can be declared as $\sum_i^M \beta_i y_i$ (as part of $coY$):
$$w^Tz_2 +  w_0 <0$$
If convex hull of $X$ and $Y$ are intersect, there must be a point $z=z_1=z_2$ (that lies in both convex hulls). This means, that $w^Tz +  w_0 <0$ and $w^T z +  w_0 >0$ must be true at the same time. But, as we see, it can't be true for any $z, w, w_0$, so that means, that if two convex hulls of two data sets intersect, these two data sets can't be  linearly separated.
\exercise
If two data sets are linearly separated, it means that $p_{MAXlikelihood}=1$ for $x>0$ (given $w^Tx=0$ for decision boundary),  We know, that 
$$p=\frac{1}{1+e^{-w^Tx}}=1$$
$$1+e^{-w^Tx}=1$$
$$e^{-w^Tx}=0$$
$$w^T \rightarrow \infty$$
We can prevent this by adding weights regularization to the loss function. That means, our algorithm still find the line, that splits all points perfectly, but posterior distribution won't be step-wise.

\exercise
If we look carefully at given data set, we can notice that all crosses are only when $x_1$ and $x_2$ have different sign, and circles, when $x_1$ and $x_2$ are the same sigh. Also, all of them are on circle with radius 1. Given that, the basis function will be:
$$\phi(x_1,x_2) = (x_1*x_2, 1)$$
Linear separator will be a line $x_1*x_2=0$

\exercise
From the task, we see, that for each point we have $bias$, $x_1$ and $x_2$. That means, that our decision boundary must be:
$$w^Tx=0$$ where $b=w_0$, which means
$$w_0*1+w_1*x_1+w_2*x_2=0$$
We have points (1; 2; 0) and (1;0;5). So we have: 
$$ \left\{
\begin{aligned}
w_0+2w_1&=0\\
w_0+5w_2 & = 0.\\
\end{aligned}
\right. $$
$$w_1=2.5w_2$$
There can be a lot of different $w_1$ and $w_2$, but let's choose $w_1=5$ and $w_2=2$.
So, $w_0$ must fulfill our equation, and must be $-10$.
Then, $w^T$ with bias will be (-10; 5;2).
\end{document}
