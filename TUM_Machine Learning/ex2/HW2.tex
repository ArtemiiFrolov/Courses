\documentclass[11pt]{article}

% ------------------------------------------------------------------------------
% This is all preamble stuff that you don't have to worry about.
% Head down to where it says "Start here"
% ------------------------------------------------------------------------------

\usepackage[margin=.8in,top=1.1in,bottom=1.1in]{geometry} % page layout
\usepackage{amsmath,amsthm,amssymb,amsfonts} % math things
\usepackage{graphicx} % include graphics
\usepackage{fancyhdr} % header customization
\usepackage{titlesec} % help with section naming

% naming sections
\titleformat{\section}{\bf}{Problem \thesection}{0.5em}{}
\newcommand{\exercise}{\section{}}

% headers
\pagestyle{fancy} 
\fancyhf{} % clear all
\fancyhead[L]{\sffamily\small Machine Learning 1 --- Homework}
\fancyhead[R]{\sffamily\small Page \thepage}
\renewcommand{\headrulewidth}{0.2pt}
\renewcommand{\footrulewidth}{0.2pt}
\markright{\hrulefill\quad}

\newcommand{\hwhead}[4]{
\begin{center}
\sffamily\large\bfseries Machine Learning Worksheet #1
\vspace{2mm} 
\normalfont

#2 -- #3 -- \texttt{#4}
\end{center}
\vspace{6mm} \hrule \vspace{4mm}
}

% ------------------------------------------------------------------------------
% Start here -- Fill in your name, imat and email
% ------------------------------------------------------------------------------

\newcommand{\name}{} %
\newcommand{\imat}{} %
\newcommand{\email}{} %

\begin{document}

% ------------------------------------------------------------------------------
% Change xx (and only xx) to the current sheet number
% ------------------------------------------------------------------------------
\hwhead{xx}{\name}{\imat}{\email}

% ------------------------------------------------------------------------------
% Fill in your solutions
% ------------------------------------------------------------------------------

\exercise % each new exercise begins with this command

Let's assume A - the person is terrorist, $\bar{A}$  is not terrorist; B - scanner identify person as terrorist, $\bar{B}$ - scanner identify person as civilian. \newline
Then, the the task can be shown as:\newline
$$ p(B\mid A) = 0.95 $$
$$p(\bar{B}\mid \bar{A}) = 0.95  \Rightarrow   p(B \mid  \bar{A} ) = 0.05 $$ 
$$ p(A) = 0.01 $$ 
Thus, 
$$p(A\mid B) = \frac{p(B\mid A)*p(A)}{p(B)} = \frac{p(B\mid A)*p(A)}{p(B\mid A)*p(A)+p(B\mid \bar{A})*p(\bar{A})} $$
$$p(A\mid B) = \frac{0.95*0.01}{0.95*0.01+0.05*0.99}=0.16$$
Answer: 0.16


\exercise
Let's assume H - we put red ball in box, T - white ball in the box. R - we took red ball from the box, W - white ball. Then, the task can be shown as:
$$p(H) = p(T) = 0.5$$
$$p(HT) = p(TH) = p(TT)= p(HH)= 0.25$$
$$p(RRR\mid HH) = 1,\quad p(RRR\mid TH) = p(RRR\mid HT) = 0.5^3 = 0.125,\quad p(RRR\mid TT) = 0$$
Thus, 
$$p(HH\mid RRR) = \frac{p(RRR\mid HH)*p(HH)}{p(RRR)}=$$
$$= \frac{p(RRR\mid HH)*p(HH)}{p(RRR\mid HH)*p(HH)+p(RRR\mid TH)*p(TH)+p(RRR\mid HT)*p(HT)+p(RRR\mid TT)*p(TT)} $$
$$p(HH\mid RRR) = \frac{1*0.25}{1*0.25+0.125*0.25+0.125*0.25+0}=0.8$$
Answer: 0.8


\exercise
Let's try to find probability of each ending on each step tossing a coin. Probability, that all ends on first step will be $0.5$, second - $0.5*0.5$, third - $0.5*0.5*0.5$ and so on. Thus, ${p}_{n}=\frac{1}{2^n} $. Then, try to find amount of coin on each step. For heads it will be always 1 coin, for tails $n-1$.
So, to find mean of amount on each step, let's find:
$$\mathbb{E}[H] = \sum_{i=1}^{\infty}1*\frac{1}{2^i}=1$$
$$\mathbb{E}[T] = \sum_{i=1}^{\infty}(i-1)*\frac{1}{2^i}=1$$
Answer: 1 heads and 1 tails


\exercise
\begin{equation}
\begin{split}
p(x) = \begin{cases}\frac{1}{b-a}, & \mbox{if } a<=x<=b \\ 0, & \mbox{elsewhere} \end{cases} \newline
\end{split}
\end{equation}

\begin{equation}
\begin{split}
\mathbb{E}[X] &= \int^{\infty}_{-\infty}{xp(x)dx} = \int^{a}_{-\infty}{0*p(x)dx} + \int^{b}_{a}{xp(x)dx} + \int^{\infty}_{b}{0*p(x)dx} \\
&= \int^{b}_{a}{x\dfrac{1}{b-a}dx} = \dfrac{1}{b-a}\int^{a}_{b}{xdx} = \dfrac{1}{b-a}(\dfrac{b^2}{2} - \dfrac{a^2}{2}) = \dfrac{a+b}{2}
\end{split}
\end{equation}

\begin{equation}
\begin{split}
Var[X] &= \mathbb{E}[X^2] - \mathbb{E}^2[X]) 
\end{split}
\end{equation}

\begin{equation}
\begin{split}
\mathbb{E}[X^2] &= \int^{\infty}_{-\infty}{x^2p(x)dx} = \int^{a}_{-\infty}{0*p(x)dx} + \int^{b}_{a}{x^2p(x)dx} + \int^{\infty}_{b}{0*p(x)dx} = \\
&= \int^{b}_{a}{x^2\dfrac{1}{b-a}dx} = \dfrac{1}{b-a}\int^{a}_{b}{x^2dx} = \dfrac{1}{b-a}(\dfrac{b^3}{2} - \dfrac{a^3}{3}) = \dfrac{a^2+ab+b^2}{2}
\end{split}
\end{equation}

\begin{equation}
\begin{split}
\mathbb{E}^2[X] = \frac{(a+b)^2}{4}
\end{split}
\end{equation}

\begin{equation}
\begin{split}
Var[X] &= \dfrac{a^2+ab+b^2}{3} - \dfrac{a^2+2ab+b^2}{4} = \dfrac{(a - b)^2}{12}
\end{split}
\end{equation}
Answer: (2) and (6)


\exercise
$$\mathbb{E}[X] = \mathbb{E}_{y}[\mathbb{E}_{x\mid y}[X]]$$
Let's proof this equation for finite cases (I tried to do it in general case, failed, and googled the decision. So I think, it will be unfair to write someone's other decision, so I post only mine, that includes only finite cases).

\begin{equation}
\begin{split}
\mathbb{E}[X] = \sum_{x}xp(x)
\end{split}
\end{equation}

\begin{equation}
\begin{split}
\mathbb{E}_{y}[\mathbb{E}_{x\mid y}[X]] &= \sum_{y}p(y)\sum_{x}xp(x\mid y) = \sum_{x}x\sum_{y}p(y)p(x\mid y)=\\
&= \sum_{x}x\sum_{y}p(y)p(x\mid y)= \sum_{x}x\sum_{y}p(y\mid x)*p(x) =\\
&= \sum_{x}xp(x)\sum_{y}p(y\mid x) = \sum_{x}xp(x) = \mathbb{E}[X] 
\end{split}
\end{equation}

\begin{equation}
\begin{split}
Var[X] &= \mathbb{E}[X^2] - \mathbb{E}^2[X] = \mathbb{E}[\mathbb{E}[x^2\mid y]]- \mathbb{E}^2[X]
\end{split}
\end{equation}

\begin{equation}
\begin{split}
Var[x\mid y] = \mathbb{E}[x^2\mid y] -\mathbb{E}^2[x\mid y]\\
 \mathbb{E}[x^2\mid y] = Var[x\mid y] + \mathbb{E}^2[x\mid y]
\end{split}
\end{equation}

using (9) and (10):
\begin{equation}
\begin{split}
Var[X] &= \mathbb{E}[Var[x\mid y] + \mathbb{E}^2[x\mid y]]- \mathbb{E}^2[X]=\\
&=\mathbb{E}[Var[x\mid y]]+\mathbb{E}[\mathbb{E}^2[x\mid y]]-\mathbb{E}^2[\mathbb{E}[x\mid y]]=\\
&= \mathbb{E}[Var[x\mid y]]+Var[\mathbb{E}[x\mid y]]
\end{split}
\end{equation}
Answer: (8) and (11)


\exercise
\begin{equation}
\begin{split}
p(\left|\frac{1}{n}\sum_{i=1}^{n}x_{i}-\mathbb{E}[x_{i}]  \right|>\epsilon)= p(\left|\frac{1}{n}(\sum_{i=1}^{n}x_{i}-\sum_{i=1}^{n}\mathbb{E}[x_{i}]) \right|>\epsilon)=\\
=p(\left|\frac{1}{n}\sum_{i=1}^{n}x_{i}-\frac{1}{n}\sum_{i=1}^{n}\mathbb{E}[x_{i}]  \right|>\epsilon)= p(\left|\frac{1}{n}\sum_{i=1}^{n}x_{i}-\mathbb{E}[\frac{1}{n}\sum_{i=1}^{n}x_{i}]  \right|>\epsilon)
\end{split}
\end{equation}
Using Chebyshev inequality:
\begin{equation}
\begin{split}
p(\left|\frac{1}{n}\sum_{i=1}^{n}x_{i}-\mathbb{E}[\frac{1}{n}\sum_{i=1}^{n}x_{i}]  \right|>\epsilon)<=\frac{Var[\frac{1}{n}\sum_{i=1}^{n}x_{i}]}{\epsilon^2} = \\
=\frac{Var[\sum_{i=1}^{n}x_{i}]]}{n^2\epsilon^2}=\frac{n*Var[x]}{n^2\epsilon^2}=\frac{Var[x]}{n\epsilon^2}
\end{split}
\end{equation}
With n$\Rightarrow\infty$, $\frac{Var[x]}{n\epsilon^2}\Rightarrow 0$, therefore $p(\left|\frac{1}{n}\sum_{i=1}^{n}x_{i}-\mathbb{E}[x_{i}]  \right|>\epsilon\Rightarrow 0)$
\end{document}
